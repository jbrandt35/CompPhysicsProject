\documentclass[a4paper,12pt]{article} % only 10 (default), 11 and 12 pt are available

% NECESSARY PACKAGES
\usepackage[utf8]{inputenc} % to be able to use non-English characters
\usepackage{amsmath}  % improve math presentation
\usepackage{amssymb}
\usepackage{xcolor}
\usepackage{float}
\usepackage{caption} % captioning package of wonders
\usepackage{url} % to incorporate clickable links
\usepackage{cite} % takes care of citations
\usepackage[final]{hyperref} % adds hyper links inside the generated PDF file

% NECESSARY FOR INCREASED CUSTOMIZATION PACKAGES
\usepackage{newfloat} % for defining new float in environments (e.g: figure and tables are floats)
\usepackage{graphicx} % takes care of graphic including machinery
\usepackage{tabularx} % extra features for tabular environment
\usepackage{array} % provides 'programmable' tables (cells can be tweaked much more)
\usepackage[export]{adjustbox} % to include boxed content other than figures (e.g: boxed equations)
\usepackage{wrapfig} % to make figures wrap around text
\usepackage{subcaption} % to make sub-figures inside of figures

% TWEAKS FOR PACKAGES
\usepackage[margin=2cm,a4paper]{geometry} % tweaks margins
\captionsetup{justification=centering} % caption justification
\captionsetup{font=small} % caption text size
\captionsetup{labelfont=bf} % caption label font emphasis (e.g: 'Figure X:')
\numberwithin{equation}{section} % equations are named according to their section (e.g: 2.1)
\numberwithin{figure}{section} % figures are named according to their section (e.g: 3.4)
\hypersetup{
    colorlinks=true,      % false: boxed links; true: colored links
    linkcolor=black,      % color of internal links
    citecolor=blue,       % color of links to bibliography
    filecolor=magenta,    % color of file links
    urlcolor=red          % color of website links
}

\usepackage{lipsum} % DELETE THIS AND BELOW \lipsum[]'s AND YOU'RE GOOD TO GO

%++++++++++++++++++++++++++++++++++++++++++++++++++++++++++++++++++++++++++++++++

% HERE GOES THE COVER PAGE SETUP
\newcommand{\hwcourse}{\text{Project Proposal}} % Title of your document
\newcommand{\hwnumber}{\text{PHYS 3266}} % Name of your study number
\newcommand{\hwdetails}{ \text{Simulating Orbital Perturbations and Inferring their Sources} \\ }
\newcommand{\hwauthor}{-Joshua Brandt- \\
                        -Paul Vollrath-  \\
                        -Chloe Fair- } % Your name or your group's names
\newcommand{\HRule}{\rule{\linewidth}{0.5mm}} % line widths in the cover page


%++++++++++++++++++++++++++++++++++++++++++++++++++++++++++++++++++++++++++++++++

\begin{document}

% COVER PAGE IS COMPILED HERE
\begin{titlepage}

\begin{center} % Center remainder of the page
% LOGO SECTION
\includegraphics[width = 8cm]{SoPBanner.jpeg}

\\[2cm]

% HEADING SECTIONS
\textsc{\LARGE Georgia Institute of Technology}\\[.5cm]
\textsc{\Large School of Physics}\\[1.5cm]

% TITLE SECTION
\HRule \\[0.4cm]
{ \huge \bfseries \hwcourse}\\ \vspace{.5cm}
{ \huge \bfseries \hwnumber}\\ \vspace{.5cm}
{ \large \bfseries \hwname}\\ \vspace{.5cm}
{ \hwdetails}\\ \vspace{.5cm}
{ \bfseries \hwdate}\\ \vspace{.5cm}
\HRule \\[1.5cm]
\end{center}

% AUTHOR SECTION
\begin{flushleft} % left oriented author section
  \centering
    \large
    \textit{Written By:}\\
    \hwauthor% Your name
\end{flushleft}
\vspace{5cm}
\makeatletter
Date: \@date
\vfill % Fill the rest of the page with white space
\makeatother
\end{titlepage}

%++++++++++++++++++++++++++++++++++++++++++++++++++++++++++++++++++++++++++++++++


\section{Research Topic \& Question}

Here we put our question


\newpage

\section{Simulation Description - We are analysing our system by way of an N-Body program, which is a tool 
that allows for simulations of multiple bodies interacting according to physical laws and specified boundary 
conditions. In this case, we are using an N-Body simulator to model the orbits of planets around the sun in our solar system.
The fundamental structure of the N-Body program is that of an interative process, which uses the laws of gravitation at 
a certain moment in time, and then uses that information to interpolate the current state to the next iteration generation. This 
process is begun with the program reading the parameters and intial states of the bodies participating in the simulation 
from a configuration JSON file, which includes reading the desired simulation duration as well as the generation time step.
The program also reads information of inputed instances of an objects class, such as each bodies mass, initial position and velocity. 
This information is then fed into the main function that starts the simulation.
After the system is initizlised, the first step in every generation in each iteration is to calculate the net force on each body. 
Beginning with the first instance of the objects class, the program then iterates to the next object instance, calculates the 
separation from the stored position attributes, and then evaluates the force on the first instance due to the second using Newton’s 
Law of Gravitation. Looping over every subsequent instance of the objects class yields the net force on the first object. This 
process is then repeated for all inputted object instances in the same manner as described before.
From the net force on each body, it is simply a matter of dividing by the mass to yield the acceleration, and then using kinematic 
equations of motion, find the position of each body a small time after. Note that in this we approximate the net force on each body 
over a small time step to be constant, which is a reasonable assumption to make given the small time step, and relatively large 
astronomical distances, as well as the fact that while in modeling the solar system the separation between planets may vary wildly, 
the primary contributor to the net force on each object is the sun, whose distance to an individual object does not vary noticeably 
over a small time interval. Coming to the end of the generation, the program updates the position and velocity attributes for each 
object in the objects class, as well as adds that generations information into a log built into each object that records position, 
velocity, and kinetic energy, for later modeling.
}

\newpage

\section{Required Input Physics}

$$\vec{F} = m \vec{a}$$

\newpage

\section{Computational Methods}

\newpage

\section{Simulation Set-up}

\newpage

\section{Quantities to Inspect}

\newpage

\bibliography{}

\end{document}
