\documentclass[a4paper,12pt]{article} % only 10 (default), 11 and 12 pt are available

% NECESSARY PACKAGES
\usepackage[utf8]{inputenc} % to be able to use non-English characters
\usepackage{amsmath}  % improve math presentation
\usepackage{amssymb}
\usepackage{xcolor}
\usepackage{float}
\usepackage{caption} % captioning package of wonders
\usepackage{url} % to incorporate clickable links
\usepackage{cite} % takes care of citations
\usepackage[final]{hyperref} % adds hyper links inside the generated PDF file

% NECESSARY FOR INCREASED CUSTOMIZATION PACKAGES
\usepackage{newfloat} % for defining new float in environments (e.g: figure and tables are floats)
\usepackage{graphicx} % takes care of graphic including machinery
\usepackage{tabularx} % extra features for tabular environment
\usepackage{array} % provides 'programmable' tables (cells can be tweaked much more)
\usepackage[export]{adjustbox} % to include boxed content other than figures (e.g: boxed equations)
\usepackage{wrapfig} % to make figures wrap around text
\usepackage{subcaption} % to make sub-figures inside of figures

% TWEAKS FOR PACKAGES
\usepackage[margin=2.54cm,a4paper]{geometry} % tweaks margins
\captionsetup{justification=centering} % caption justification
\captionsetup{font=12pt} % caption text size
\captionsetup{labelfont=bf} % caption label font emphasis (e.g: 'Figure X:')
\numberwithin{equation}{section} % equations are named according to their section (e.g: 2.1)
\numberwithin{figure}{section} % figures are named according to their section (e.g: 3.4)
\hypersetup{
    colorlinks=true,      % false: boxed links; true: colored links
    linkcolor=black,      % color of internal links
    citecolor=blue,       % color of links to bibliography
    filecolor=magenta,    % color of file links
    urlcolor=red          % color of website links
}

\usepackage{lipsum} % DELETE THIS AND BELOW \lipsum[]'s AND YOU'RE GOOD TO GO

%++++++++++++++++++++++++++++++++++++++++++++++++++++++++++++++++++++++++++++++++

% HERE GOES THE COVER PAGE SETUP
\newcommand{\hwcourse}{\text{Progress Report}} % Title of your document
\newcommand{\hwnumber}{\text{PHYS 3266}} % Name of your study number
\newcommand{\hwdetails}{ \text{Simulating Orbital Perturbations and Inferring their Sources} \\ }
\newcommand{\hwauthor}{-Joshua Brandt- \\
                        -Paul Vollrath-  \\
                        -Chloe Fair- } % Your name or your group's names
\newcommand{\HRule}{\rule{\linewidth}{0.5mm}} % line widths in the cover page


%++++++++++++++++++++++++++++++++++++++++++++++++++++++++++++++++++++++++++++++++

\begin{document}

% COVER PAGE IS COMPILED HERE
\begin{titlepage}

\begin{center} % Center remainder of the page
% LOGO SECTION
\includegraphics[width = 8cm]{SoPBanner.jpeg}

\\[2cm]

% HEADING SECTIONS
\textsc{\LARGE Georgia Institute of Technology}\\[.5cm]
\textsc{\Large School of Physics}\\[1.5cm]

% TITLE SECTION
\HRule \\[0.4cm]
{ \huge \bfseries \hwcourse}\\ \vspace{.5cm}
{ \huge \bfseries \hwnumber}\\ \vspace{.5cm}
{ \large \bfseries \hwname}\\ \vspace{.5cm}
{ \hwdetails}\\ \vspace{.5cm}
{ \bfseries \hwdate}\\ \vspace{.5cm}
\HRule \\[1.5cm]
\end{center}

% AUTHOR SECTION
\begin{flushleft} % left oriented author section
  \centering
    \large
    \textit{Written By:}\\
    \hwauthor% Your name
\end{flushleft}
\vspace{5cm}
\makeatletter
Date: \@date
\vfill % Fill the rest of the page with white space
\makeatother
\end{titlepage}

%++++++++++++++++++++++++++++++++++++++++++++++++++++++++++++++++++++++++++++++++


\section{Research Topic \& Question}

Subfield: Astrophysics (Celestial Mechanics) \newline
Question: Given a deviation from a planetary Keplerian orbital trajectory, can we determine properties of an intervening planet? \newline
The goal of our project is to create a computational simulation of a method that was used to predict the location and properties of undiscovered planets - which ultimatley led to the predictions and discovieries and Neptune and Pluto, and continues to create wonder about more potential planets in the Solar System. Our goal is to take in data on the orbits of a set of planets and use and N-body simulator to guess where a planet would need to be to create observed orbital perturbations.

\section{Current Setup}



\section{Current Status}



\section{Difficulties}




\section{Figures}



\section{Quantities to Inspect}



\end{document}
